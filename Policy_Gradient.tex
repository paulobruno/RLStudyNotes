\documentclass{article}
\usepackage[utf8]{inputenc}
\usepackage[brazil]{babel}
\usepackage[a4paper, top=2.5cm, bottom=2.5cm, left=2cm, right=2.5cm]{geometry}
\usepackage{makecell}

\usepackage{pgf}
\usepackage{tikz}
\usetikzlibrary{arrows,automata}

\usepackage{graphicx}
\graphicspath{{figures/}}

\usepackage{amsmath,amssymb}
\DeclareMathOperator*{\argmax}{argmax}

% PB: redefinir maketitle
\makeatletter
\def\@maketitle
{
    \begin{flushleft}
        \let \footnote \thanks
        {\Large \textbf{\@title} \par}
        \vskip 1em
        {\large \textbf{\@author} \par}
        \vskip 1em
        {\large \textit{\@date}}
    \end{flushleft}
    \par
    \vskip 1.5em
}
\makeatother

\title{Aprendizado por Reforço}
\author{Aula  - Policy Gradient}
\date{Paulo Bruno de Sousa Serafim - Outubro 2019}

\begin{document}

\maketitle

\section{Preferências}

    \subsection{\emph{Softmax}}
        
    \subsection{Adição de um \emph{Baseline}}
    
    \subsection{Gradiente Ascendente}
    
\section{Aproximação de políticas}

    \subsection{\emph{Policy Gradient Theorem}}
    
\section{\emph{REINFORCE}}

\section{Modelos \emph{Actor-Critic}}

\section{Problemas Contínuos}

\section{Comparação com Diferenças Temporais}

    \subsection{Vantagens x Desvantagens}
    
\end{document}