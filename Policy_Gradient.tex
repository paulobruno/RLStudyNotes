\documentclass{article}
\usepackage[utf8]{inputenc}
\usepackage[brazil]{babel}
\usepackage[a4paper, top=2.5cm, bottom=2.5cm, left=2cm, right=2.5cm]{geometry}
\usepackage{makecell}

\usepackage{pgf}
\usepackage{tikz}
\usetikzlibrary{arrows,automata}

\usepackage{graphicx}
\graphicspath{{figures/}}

\usepackage{amsmath,amssymb}
\DeclareMathOperator*{\argmax}{argmax}

% PB: redefinir maketitle
\makeatletter
\def\@maketitle
{
    \begin{flushleft}
        \let \footnote \thanks
        {\Large \textbf{\@title} \par}
        \vskip 1em
        {\large \textbf{\@author} \par}
        \vskip 1em
        {\large \textit{\@date}}
    \end{flushleft}
    \par
    \vskip 1.5em
}
\makeatother

\title{Aprendizado por Reforço}
\author{Aula  - Policy Gradient}
\date{Paulo Bruno de Sousa Serafim - Outubro 2019}

\begin{document}

\maketitle

\section{Preferências}

    \subsection{\emph{Softmax}}
        
        \begin{equation}
            f(x_i) = \frac{e^{x_i}}{\sum_{j=1}^{k}e^{x_j}}
        \end{equation}
    
        Uma propriedade interessante da função \emph{softmax} é que adicionar ou subtrair um valor constante $B$ de todos os elementos não modifica o valor de $f(x_i)$, i.e.:
        
        \begin{equation}
        \label{eq:softmax-constant}
            \frac{e^{x_i + B}}{\sum_{j=1}^{k}e^{x_j + B}} = \frac{e^{x_i}}{\sum_{j=1}^{k}e^{x_j}}
        \end{equation}
    
        Exercício: comparar se essa propriedade existe em uma média aritmética.
        
        Exercício: provar o resultado da equação \eqref{eq:softmax-constant}

        \begin{equation}
            Pr\{A_t=a\} \ \dot{=} \  \frac{e^{H_t(a)}}{\sum_{b=1}^{k}e^{H_t(b)}} \ \dot{=} \ \pi_t(a)
        \end{equation}
    
        \begin{equation}
            H_{t+1}(A_t) \ \dot{=} \ H_t(A_t) + \alpha R_t
        \end{equation}

    \subsection{Adição de um \emph{Baseline}}
    
        \begin{equation}
            H_{t+1}(A_t) \ \dot{=} \ H_t(A_t) + \alpha (R_t - B_t)
        \end{equation}
    
    \subsection{Gradient Ascendente}
    
        \begin{equation}
            H_{t+1}(a) \ \dot{=} \ H_t(a) + \alpha \frac{\partial \mathbb{E}[R_t]}{\partial H_t(a)}
        \end{equation}
        
        onde:
        
        \begin{equation}
            \mathbb{E}[R_t] = \sum_x \pi_t(x) q_*(x)
        \end{equation}
    
\section{Aproximação de políticas}

    \subsection{\emph{Policy Gradient Theorem}}
    
\section{\emph{REINFORCE}}

\section{Modelos \emph{Actor-Critic}}

\section{Problemas Contínuos}

\section{Comparação com Diferenças Temporais}

    \subsection{Vantagens x Desvantagens}
    
\end{document}