\documentclass{article}
\usepackage[utf8]{inputenc}
\usepackage[brazil]{babel}
\usepackage[a4paper, top=2.5cm, bottom=2.5cm, left=2cm, right=2.5cm]{geometry}

\usepackage{amsmath,amssymb}
\DeclareMathOperator*{\argmax}{argmax}

\usepackage{pgf}
\usepackage{tikz}
\usetikzlibrary{arrows,automata,plotmarks}
\usepackage{pgfplots}
\pgfplotsset{compat=1.15}

\tikzset{
    % node styles
    state-node/.style={
        fill=none, shape=circle, draw=black, thick, text=black, minimum size=0.6cm},
    action-node/.style={
        fill=black, draw=none, text=white, shape=circle, inner sep=0.05cm, minimum size=0.2cm},
    reward-node/.style={
        fill=none, draw=black, text=black},
    hidden-node/.style={
        fill=none, draw=none, text=white, shape=circle, inner sep=0,outer sep=0, minimum size=0.0cm},
    % label styles
    action-label/.style={
        shape=circle, text=white, draw=none, fill=black, inner sep=0.05cm, minimum size=0.2cm, align=center, yshift=0.0cm, anchor=center},
    reward-label/.style={
        shape=rectangle, text=black, draw=black, fill=white, minimum size=0.5cm, align=center, yshift=0.0cm, anchor=center},
    hidden-edge/.style={
        text=white, draw=none, fill=none, inner sep=0,outer sep=0, minimum size=0.0cm},
}


% PB: redefinir maketitle
\makeatletter
\def\@maketitle
{
    \begin{flushleft}
        \let \footnote \thanks
        {\Large \textbf{\@title} \par}
        \vskip 1em
        {\large \textbf{\@author} \par}
        \vskip 1em
        {\large \textit{\@date}}
    \end{flushleft}
    \par
    \vskip 1.5em
}
\makeatother


\title{Notas de estudo - Aprendizado por Reforço}
\author{Parte 06 - Diferenças Temporais}
\date{Paulo Bruno de Sousa Serafim - Out/19 - Jan/20}


\begin{document}

\maketitle

    \section{Introdução}
    
        \subsection{Equação de atualização}
        
            \begin{equation}
                V(S_t) \ \leftarrow V(S_t) + \alpha \Big[ R_{t+1} + \gamma V(S_{t+1}) - V(S_t) \Big]
            \end{equation}
        
            TD-error:
            
            \begin{equation}
                \delta_t \ \dot{=} \ R_{t+1} + \gamma V(S_{t+1}) - V(S_t)
            \end{equation}
        
    \section{Sarsa}
    
        \textit{On-policy}
    
        Nesse caso utilizamos a mesma política estimada para escolher a ação do estado seguinte. Para simplificar a notação, omitimos a política $\pi$ da equação de atualização da função de valor-ação do \textit{Sarsa}:
    
        \begin{equation}
            Q(S_t, A_t) \leftarrow Q(S_t, A_t) + \alpha \Big[ R_{t+1} + \gamma Q(S_{t+1}, A_{t+1}) - Q(S_t, A_t) \Big]
        \end{equation}
        
    \section{Q-Learning}
    
        \textit{Off-policy}
        
        A política estimada não é utilizada para escolher a ação no estado seguinte. Em especial, se for utilizada uma política gulosa, temos o algoritmo \textit{Q-Learning}:
    
        \begin{equation}
            Q(S_t, A_t) \leftarrow Q(S_t, A_t) + \alpha \Big[ R_{t+1} + \gamma \max_{a} Q(S_{t+1}, a) - Q(S_t, A_t) \Big]
        \end{equation}
    
    \section{Expected Sarsa}
    
        \begin{equation}
        \begin{split}
            Q(S_t, A_t) & \leftarrow Q(S_t, A_t) + \alpha \Big[ R_{t+1} + \gamma \mathbb{E} \big[ Q(S_{t+1}, A_{t+1}) \mid S_{t+1} \big] - Q(S_t, A_t) \Big] \\
            & \leftarrow Q(S_t, A_t) + \alpha \Big[ R_{t+1} + \gamma \sum_a \pi(a \mid S_{t+1}) Q(S_{t+1},a) - Q(S_t, A_t) \Big]
        \end{split}
        \end{equation}
        
    \section{Double Learning}
    
        \begin{equation}
        \begin{split}
            Q_{1}(S_t, A_t) \leftarrow Q_{1}(S_t, A_t) + \alpha \Big[ R_{t+1} + \gamma Q_{2}(S_{t+1}, \argmax_a Q_{1}(S_{t+1}, a)) - Q_{1}(S_t, A_t) \Big] \\
            Q_{2}(S_t, A_t) \leftarrow Q_{2}(S_t, A_t) + \alpha \Big[ R_{t+1} + \gamma Q_{1}(S_{t+1}, \argmax_a Q_{2}(S_{t+1}, a)) - Q_{2}(S_t, A_t) \Big]
        \end{split}
        \end{equation}
    
    \section{Exemplo}
    
        \begin{center}
        \begin{tikzpicture}[node distance=2.0cm]
            \tikzstyle{tile-node}=[fill=none, shape=circle, draw=black, thick, text=black, minimum size=0.8cm, inner sep=1]
            \tikzstyle{terminal-node}=[fill=gray, shape=circle, draw=black, thick, text=white, minimum size=0.8cm, inner sep=1]
        
            \node[tile-node] (A1) {};
            \node[tile-node] (A2) [above of=A1] {};
            \node[tile-node] (A3) [above of=A2] {};
            \node[tile-node] (B1) [right of=A1] {};
            \node[hidden-node] (B2) [above of=B1] {};
            \node[tile-node] (B3) [above of=B2] {};
            \node[tile-node] (C1) [right of=B1] {};
            \node[tile-node] (C2) [above of=C1] {};
            \node[tile-node] (C3) [above of=C2] {};
            \node[tile-node] (D1) [right of=C1] {};
            \node[terminal-node, fill=gray, text=white] (D2) [above of=D1] {$-5$};
            \node[terminal-node, fill=gray, text=white] (D3) [above of=D2] {$5$};
            
            \draw[->, loop left, min distance=1.0cm] (A1) to node {$0.0$} (A1);
            \draw[->, loop below, min distance=1.0cm] (A1) to node {$0.0$} (A1);
            \draw[->, transform canvas={xshift=-0.1cm}] (A1) to node[left] {$0.0$} (A2);
            \draw[->, transform canvas={yshift=0.1cm}] (A1) to node[above] {$0.0$} (B1);
            
            \draw[->, loop left, min distance=1.0cm] (A2) to node {$0.0$} (A2);
            \draw[->, transform canvas={xshift=0.1cm}] (A2) to node[right] {$0.0$} (A1);
            \draw[->, transform canvas={xshift=-0.1cm}] (A2) to node[left] {$0.0$} (A3);
            \draw[->, loop right, min distance=1.0cm] (A2) to node {$0.0$} (A2);
            
            \draw[->, loop left, min distance=1.0cm] (A3) to node {$0.0$} (A3);
            \draw[->, transform canvas={xshift=0.1cm}] (A3) to node[right] {$0.0$} (A2);
            \draw[->, transform canvas={yshift=0.1cm}] (A3) to node[above] {$0.0$} (B3);
            \draw[->, loop above, min distance=1.0cm] (A3) to node {$0.0$} (A3);
            
            \draw[->, transform canvas={yshift=-0.1cm}] (B1) to node[below] {$0.0$} (A1);
            \draw[->, transform canvas={yshift=0.1cm}] (B1) to node[above] {$0.0$} (C1);
            \draw[->, loop below, min distance=1.0cm] (B1) to node {$0.0$} (B1);
            \draw[->, loop above, min distance=1.0cm] (B1) to node {$0.0$} (B1);
            
            \draw[->, transform canvas={yshift=-0.1cm}] (B3) to node[below] {$0.0$} (A3);
            \draw[->, transform canvas={yshift=0.1cm}] (B3) to node[above] {$0.0$} (C3);
            \draw[->, loop below, min distance=1.0cm] (B3) to node {$0.0$} (B3);
            \draw[->, loop above, min distance=1.0cm] (B3) to node {$0.0$} (B3);
            
            \draw[->, transform canvas={yshift=-0.1cm}] (C1) to node[below] {$0.0$} (B1);
            \draw[->, transform canvas={yshift=0.1cm}] (C1) to node[above] {$0.0$} (D1);
            \draw[->, loop below, min distance=1.0cm] (C1) to node {$0.0$} (C1);
            \draw[->, transform canvas={xshift=-0.1cm}] (C1) to node[left] {$0.0$} (C2);
            
            \draw[->, loop left, min distance=1.0cm] (C2) to node {$0.0$} (C2);
            \draw[->, transform canvas={xshift=0.1cm}] (C2) to node[right] {$0.0$} (C1);
            \draw[->, transform canvas={xshift=-0.1cm}] (C2) to node[left] {$0.0$} (C3);
            \draw[->, transform canvas={yshift=0.1cm}] (C2) to node[above] {$0.0$} (D2);
            
            \draw[->, transform canvas={yshift=-0.1cm}] (C3) to node[below] {$0.0$} (B3);
            \draw[->, transform canvas={xshift=0.1cm}] (C3) to node[right] {$0.0$} (C2);
            \draw[->, transform canvas={yshift=0.1cm}] (C3) to node[above] {$0.0$} (D3);
            \draw[->, loop above, min distance=1.0cm] (C3) to node {$0.0$} (C3);
            
            \draw[->, transform canvas={yshift=-0.1cm}] (D1) to node[below] {$0.0$} (C1);
            \draw[->, transform canvas={xshift=-0.1cm}] (D1) to node[left] {$0.0$} (D2);
            \draw[->, loop below, min distance=1.0cm] (D1) to node {$0.0$} (D1);
            \draw[->, loop right, min distance=1.0cm] (D1) to node {$0.0$} (D1);
        \end{tikzpicture}
        \end{center}
    
\end{document}