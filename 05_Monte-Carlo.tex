\documentclass{article}
\usepackage[utf8]{inputenc}
\usepackage[brazil]{babel}
\usepackage[a4paper, top=2.5cm, bottom=2.5cm, left=2cm, right=2.5cm]{geometry}
\usepackage{makecell}

\usepackage{pgf}
\usepackage{tikz}
\usetikzlibrary{arrows,automata}

\usepackage{graphicx}
\graphicspath{{figures/}}

% PB: redefinir maketitle
\makeatletter
\def\@maketitle
{
    \begin{flushleft}
        \let \footnote \thanks
        {\Large \textbf{\@title} \par}
        \vskip 1em
        {\large \textbf{\@author} \par}
        \vskip 1em
        {\large \textit{\@date}}
    \end{flushleft}
    \par
    \vskip 1.5em
}
\makeatother


\title{Notas de estudo - Aprendizado por Reforço}
\author{Parte 05 - Métodos de Monte-Carlo}
\date{Paulo Bruno de Sousa Serafim - Out/19 - Jan/20}

\begin{document}

\maketitle

\section{Introdução}

    Mostrar grafos de MDP's sem probabilidades e com próximos estados desconhecidos.
    
    \subsection{Monte-Carlo ``primeira-visita''}
    
    \subsection{Monte-Carlo ``cada-visita''}

\section{Estimativas dos valores das ações}

    Explicar o que é \textit{exploring starts}

\section{Controle Monte-Carlo}

\section{Métodos \textit{On-policy} vs. \textit{Off-policy}}

\section{Árvore de Busca de Monte-Carlo (MCTS)}

    \subsection{Técnicas de \textit{Rollout}}
    
    \subsection{Algoritmo}

\end{document}