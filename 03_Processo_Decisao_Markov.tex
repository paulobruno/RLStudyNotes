\documentclass{article}
\usepackage[utf8]{inputenc}
\usepackage[brazil]{babel}
\usepackage[a4paper, top=2.5cm, bottom=2.5cm, left=2cm, right=2.5cm]{geometry}
\usepackage{makecell}
\usepackage{indentfirst}

\usepackage{pgf}
\usepackage{tikz}
\usetikzlibrary{arrows,automata}

\usepackage{graphicx}
\graphicspath{{figures/}}

% PB: redefinir maketitle
\makeatletter
\def\@maketitle
{
    \begin{flushleft}
        \let \footnote \thanks
        {\Large \textbf{\@title} \par}
        \vskip 1em
        {\large \textbf{\@author} \par}
        \vskip 1em
        {\large \textit{\@date}}
    \end{flushleft}
    \par
    \vskip 1.5em
}
\makeatother

\tikzset{
    % node styles
    state-node/.style={
        fill=none, shape=circle, draw=black, thick, text=black, minimum size=0.6cm},
    action-node/.style={
        fill=black, draw=none, text=white, shape=circle, inner sep=0.05cm, minimum size=0.2cm},
    reward-node/.style={
        fill=none, draw=black, text=black},
    hidden-node/.style={
        fill=none, draw=none, text=white, shape=circle, inner sep=0,outer sep=0, minimum size=0.0cm},
    % label styles
    action-label/.style={
        shape=circle, text=white, draw=none, fill=black, inner sep=0.05cm, minimum size=0.2cm, align=center, yshift=0.0cm, anchor=center},
    reward-label/.style={
        shape=rectangle, text=black, draw=black, fill=white, minimum size=0.5cm, align=center, yshift=0.0cm, anchor=center},
    hidden-edge/.style={
        text=white, draw=none, fill=none, inner sep=0,outer sep=0, minimum size=0.0cm},
}


\title{Aprendizado por Reforço}
\author{Aula 03 - Processo de Decisão de Markov (MDP)}
\date{Paulo Bruno de Sousa Serafim - Outubro 2019}

\begin{document}

\maketitle

\section{Caracterização da interação agente-ambiente}

    \subsection{Definções}
        
        \subsubsection{Agente}
        \subsubsection{Ambiente}
        \subsubsection{Estado}
        \subsubsection{Ação}
        \subsubsection{Recompensa}
        \subsubsection{Trajetória}
        
    \subsection{Diagrama de interação}
    
        \textcolor{red}{a recompensa $R_t+1$ deve ser recebida antes de $S_t$}
    
        \begin{center}
        \begin{tikzpicture}[->,>=latex, auto, node distance=2.0cm, very thick, font=\small]
            \tikzstyle{rect-node}=[fill=none,shape=rectangle,draw=black,text=black,rounded corners=0.1cm, inner sep=0.4cm]
            \tikzstyle{hidden-node}=[fill=none, draw=none, text=black, shape=rectangle, inner sep=0,outer sep=0.05cm, minimum height=0.75cm]
            
            \node[rect-node]  (Agent)                                     {\normalsize Agent};
            \node[rect-node]  (Env)     [below of=Agent]                  {\normalsize Environment};
            \node[hidden-node] (Hidden)  [left of=Env, xshift=-0.5cm]     {};
            \node[hidden-node] (UpHid)   [above of=Hidden, yshift=-1.0cm] {};
            \node[hidden-node] (DownHid) [below of=Hidden, yshift=1.0cm]  {};
            
            \draw[thick, transform canvas={yshift=0.25cm}] (Env) to node[above] 
                {$R_{t+1}$} (Hidden);
            \draw[ultra thick, transform canvas={yshift=-0.25cm}] (Env) to node[above] 
                {$S_{t+1}$} (Hidden);
            \draw[ultra thick] (Hidden.255) to node[left, pos=1.0, yshift=1.25cm, align=center] 
                {state\\$S_t$}  ++(-1.5,0) |- (Agent.165);
            \draw[thick] (Hidden.105) to node[right, pos=1.0, yshift=0.75cm, align=center] 
                {reward\\$R_t$} ++(-1.0,0) |- (Agent.195);
            \draw[ultra thick] (Agent) to node[below right, pos=1.0, yshift=-0.5cm, align=center] 
                {action\\$A_t$} ++(3.5,0)  |- (Env);
            \draw[-, thick, dashed] (UpHid) to node {} (DownHid);
        \end{tikzpicture}
        \end{center}
    
\section{Definição de MDP}

    Esse tipo de interação apresentado acima é formalizado matematicamente como um \emph{Processo de decisão de Markov} (MDP). \textcolor{red}{Os papéis do agente e ambiente são diferentes}.

    \subsection{Elementos de um MDP}

        \subsubsection{Estado}
        
            \begin{tikzpicture}
                \node[state-node] (R) {$s_1$};
            \end{tikzpicture}
            
            Nota: Estado vs Observação: 
            
        \subsubsection{Ação}
        
            \begin{tikzpicture}
                \node[action-node] (R) {$a_1$};
            \end{tikzpicture}
            
        \subsubsection{\textcolor{red}{Dinâmica / Probabilidade de Transição de estado}}
        
            \textcolor{red}{mostrar os dois tipos de notação possíveis $p(s', r \mid s, a)$ e $p(s, a, r, s')$. A primeira "chegar em s' recebendo r, dado que está em s e executou ação a". A segunda "está em s, executa a, recebe r e chega em s'.}
    
            $p_1$
            
        \subsubsection{Recompensa}
    
            \begin{tikzpicture}
                \node[reward-node] (R) {$r_1$};
            \end{tikzpicture}
            
    \subsection{Diagrama de interação mais adequado aos casos reais}
    
        
    
    \subsection{Características de um MDP}
    
        \subsubsection{MDP's finitos e infinitos}
        
        \subsubsection{Propriedade de Markov}
        
        \subsubsection{Cadeia de Markov}
    
\section{Visão gráfica de um MDP}

    \begin{tikzpicture}[->,>=stealth',shorten >=1pt,auto,node distance=2.8cm, semithick]
      \tikzstyle{every state}=[fill=blue,draw=none,text=white]
    
      \node[initial,state] (A)                    {$s_1$};
      \node[state]         (B) [above right of=A] {$s_2$};
      \node[state]         (D) [below right of=A] {$s_3$};
      \node[state]         (C) [below right of=B] {$s_4$};
      \node[state]         (E) [right of=C]       {$s_5$};
    
      \path (A) edge              node {$a_1$} (B)
                edge              node {$a_2$} (C)
            (B) edge [loop above] node {$a_3$} (B)
                edge              node {$a_4$} (C)
            (C) edge              node {$a_5$} (D)
                edge [bend left]  node {$a_6$} (E)
            (D) edge [loop below] node {$a_7$} (D)
                edge              node {$a_8$} (A)
            (E) edge [bend left]  node {$a_9$} (D);
    \end{tikzpicture}


    
    \begin{center}
    \begin{tikzpicture}[->,>=stealth', auto, node distance=4.0cm, thick]
        \node[state-node]  (S1) {$s_1$};
        \node[state-node]  (S2) [below of=S1, xshift=-3.0cm] {$s_2$};
        \node[state-node]  (S3) [below of=S1, xshift= 3.0cm] {$s_3$};
        \node[hidden-node] (A1) [above of=S2, yshift=-1.5cm] {};
        \node[hidden-node] (A2) [below of=S1, yshift=2.5cm]  {};
        
        \draw[bend right=20,-] (S1) to node[action-label]           {$a_1$} (A1);
        \draw[-]               (S1) to node[action-label, pos=0.4]  {$a_2$} (A2);
        \draw[bend left=40]    (S1) to node[action-label, pos=0.25] {$a_3$} (S3);
        
        \draw[bend right=30]   (A1) to node[left, pos=0.2]  {$p_1$} (S2);
        \draw[bend left=30]    (A1) to node[right, pos=0.2] {$p_2$} (S2);
        \draw[bend left=40]    (A2) to node[left, pos=0.3]  {$p_3$} (S2);
        \draw[bend right=40]   (A2) to node[right, pos=0.3] {$p_4$} (S3);
        
        \draw[hidden-edge, bend right=30] (A1) to node[reward-label]           {$r_1$} (S2);
        \draw[hidden-edge, bend left=30]  (A1) to node[reward-label]           {$r_2$} (S2);
        \draw[hidden-edge, bend left=40]  (A2) to node[reward-label]           {$r_3$} (S2);
        \draw[hidden-edge, bend right=40] (A2) to node[reward-label]           {$r_4$} (S3);
        \draw[hidden-edge, bend left=40]  (S1) to node[reward-label, pos=0.65] {$r_5$} (S3);
    \end{tikzpicture}
    \end{center}

\section{Exemplo de como representar um problema via MDP}

    \textcolor{red}{Escadas de Hogwarts}

\section{Objetivos}

    \subsection{Hipótese da recompensa}

    \subsection{Retorno esperado}
    
    \subsection{Desconto}
    
        Propósito intuitivo: recompensa imediatas são mais importantes do que as futuras.
        
        Propósito formal: transformar retornos infinitos em finitos.
    
\section{Política}

    Definição de política.

    \section{Política ótima}
    

\section{Funções de valor}
    
    \subsection{Função estado-valor}
    
    \subsection{Função ação-valor}
    
    \subsection{Otimalidade das funções de valor}
    
        \subsubsection{Valor estado}
        
        \subsubsection{Valor ação}
        
        \subsubsection{Equações de Bellman}
    
\section{Processo de Decisão de Markov Parcialmente Observável}

\end{document}