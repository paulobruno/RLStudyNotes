\documentclass{article}
\usepackage[utf8]{inputenc}
\usepackage[brazil]{babel}
\usepackage[a4paper, top=2.5cm, bottom=2.5cm, left=2cm, right=2.5cm]{geometry}
\usepackage{makecell}

\usepackage{pgf}
\usepackage{tikz}
\usetikzlibrary{arrows,automata}

\usepackage{graphicx}
\graphicspath{{figures/}}

\usepackage{amsmath,amssymb}
\DeclareMathOperator*{\argmax}{argmax} % thin space, limits underneath in displays

% PB: redefinir maketitle
\makeatletter
\def\@maketitle
{
    \begin{flushleft}
        \let \footnote \thanks
        {\Large \textbf{\@title} \par}
        \vskip 1em
        {\large \@author \par}
        \vskip 1em
        {\large \textit{\@date}}
    \end{flushleft}
    \par
    \vskip 1.5em
}
\makeatother

\title{Aprendizado por Reforço \vskip .5em Notas de aula}
\author{Paulo Bruno de Sousa Serafim}
\date{2019}

\begin{document}

\maketitle

\section{Introdução}

    \subsection{Equação de atualização}
    
\section{Sarsa}

    \begin{equation}
        Q(S_t, A_t) \leftarrow Q(S_t, A_t) + \alpha \left[ R_{t+1} + \gamma Q(S_{t+1}, A_{t+1}) - Q(S_t, A_t) \right]
    \end{equation}
    
\section{Q-Learning}

    \begin{equation}
        Q(S_t, A_t) \leftarrow Q(S_t, A_t) + \alpha \left[ R_{t+1} + \gamma \max_{a} Q(S_{t+1}, a) - Q(S_t, A_t) \right]
    \end{equation}

\section{Expected Sarsa}

    \begin{equation}
        Q(S_t, A_t) \leftarrow Q(S_t, A_t) + \alpha \left[ R_{t+1} + \gamma \mathbb{E}[Q(S_{t+1}, A_{t+1} \mid S_{t+1}] - Q(S_t, A_t) \right]
    \end{equation}
    
\section{Double Learning}

    \begin{equation}
        Q_{1}(S_t, A_t) \leftarrow Q_{1}(S_t, A_t) + \alpha \left[ R_{t+1} + \gamma Q_{2}(S_{t+1}, \argmax_a Q_{1}(S_{t+1}, a)) - Q_{1}(S_t, A_t) \right]
    \end{equation}
    
\end{document}