\documentclass{article}
\usepackage[utf8]{inputenc}
\usepackage[brazil]{babel}
\usepackage[a4paper, top=2cm, bottom=2.5cm, left=3cm, right=3cm]{geometry}
\usepackage{makecell}

\usepackage{pgf}
\usepackage{tikz}
\usetikzlibrary{arrows,automata}

\usepackage{graphicx}
\graphicspath{{figures/}}

\usepackage{amsmath,amssymb}
\DeclareMathOperator*{\argmax}{argmax}

% PB: redefinir maketitle
\makeatletter
\def\@maketitle
{
    \begin{flushleft}
        \let \footnote \thanks
        {\Large \textbf{\@title} \par}
        \vskip 1em
        {\large \textbf{\@author} \par}
        \vskip 1em
        {\large \textit{\@date}}
    \end{flushleft}
    \par
    \vskip 1.5em
}
\makeatother

\title{Aprendizado por Reforço}
\author{Aula 02 - \textit{Multi-Armed bandit}}
\date{Paulo Bruno S. Serafim - Outubro 2019}

\begin{document}

\maketitle

\section{\textit{Multi-armed bandit}}

    \begin{tikzpicture}[-,>=stealth',shorten >=1pt,auto,node distance=1.5cm]
        \tikzstyle{state}=[fill=none, shape=circle, draw=black, thick, text=black]
        \tikzstyle{hidden-state}=[fill=none, draw=black, text=black]
        \tikzstyle{action-node}=[fill=black, draw=none, text=black, shape=circle, ,inner sep=0,outer sep=0, minimum size=0.1cm]
        
        \node[state] (S1) {s};
        \node[action-node]  (A1) [below of=S1, xshift=-2.0cm] {};
        \node[action-node]  (A2) [below of=S1, xshift=2.0cm]  {};
        \node[hidden-state] (R1) [below of=A1, xshift=-1.0cm] {$r_1$};
        \node[hidden-state] (R2) [below of=A1, xshift=0.0cm]  {$r_2$};
        \node[hidden-state] (R3) [below of=A1, xshift=1.0cm]  {$r_3$};
        \node[hidden-state] (R4) [below of=A2, xshift=0.0cm]  {$r_4$};
        
        \draw[bend right=40] (S1) to node[left]  {$a_1$} (A1);
        \draw[bend left=40]  (S1) to node[right] {$a_2$} (A2);
        \draw[bend right]    (A1) to node[left]  {$p_1$} (R1);
        \draw                (A1) to node[left]  {$p_2$} (R2);
        \draw[bend left]     (A1) to node[right] {$p_3$} (R3);
        \draw                (A2) to node[right] {$p_4$} (R4);
    \end{tikzpicture}

    \subsection{Descrição informal}
    
    \subsection{Formalização}

\section{Dilema \textit{``Exploration vs. Exploitation''}}

    \subsection{Definições}
    
        \subsection{\textit{``Exploration''} = Prospecção}
        
        \subsection{\textit{``Exploitation''} = Exploração}
    
    \subsection{Estratégias}
    
        \subsubsection{Estratégias $\epsilon$}
        
            \begin{itemize}
                \item $epsilon$-greedy
                \item $epsilon$-first
                \item $epsilon$-decay
            \end{itemize}
        
        \subsubsection{Índice de Gittins}

\section{Função valor-ação}

    Função valor-ação ótima:

    \begin{equation}
        q_*(a) \ \dot{=} \ \mathbb{E}[R_t \mid A_t = a]
    \end{equation}

    Escolha gulosa:

    \begin{equation}
        A_t \ \dot{=} \ \argmax_a Q_t(a)
    \end{equation}
    
    \begin{center}
    \begin{tikzpicture}[->,>=stealth',shorten >=1pt,auto,node distance=2.8cm, semithick]
      \tikzstyle{initial-state}=[fill=green,shape=circle,draw=none,text=white]
      \tikzstyle{second-state}=[fill=red,shape=circle,draw=none,text=white]
      \tikzstyle{third-state}=[fill=blue,shape=circle,draw=none,text=white]
        
        \node[initial-state] (A) {$s_1$};
        \node[second-state]  (B) [below of=A, xshift=-4.5cm] {$s_2$};
        \node[second-state]  (C) [below of=A, xshift=-1.5cm] {$s_2$};
        \node[second-state]  (D) [below of=A, xshift= 1.5cm] {$s_2$};
        \node[third-state]   (E) [below of=A, xshift= 4.5cm] {$s_3$};
        
        \path 
            (A) edge[bend right=5] node[left] {$a_1, p_{1}$} (B)
                edge[bend right=5] node[left] {$a_1, p_{2}$} (C)
                edge[bend right=5] node[left] {$a_2, p_{3}$} (D)
                edge[bend right=5] node[left] {$a_3, p_{4}$} (E)
            (B) edge[bend right=5] node[right] {$r_1$} (A)
            (C) edge[bend right=5] node[right] {$r_2$} (A)
            (D) edge[bend right=5] node[right] {$r_3$} (A)
            (E) edge[bend right=5] node[right] {$r_4$} (A);
    \end{tikzpicture}
    \end{center}

    \subsection{Definição}
    
    \subsection{Versão incremental}
        
        \begin{equation}
            Q_n \ \dot{=} \ \frac{R_1 + R_2 + \cdots + R_{n-1}}{n - 1}
        \end{equation}
        
        \begin{equation}
        \begin{split}
            Q_{n+1} & \ \dot{=} \ \frac{1}{n} \sum_{i=1}^{n} R_i \\
            & = \ Q_n + \frac{1}{n} \Big[ R_n - Q_n \Big]
        \end{split}
        \end{equation}
        
        Ver no livro a derivação dessa equação (eq. 2.3).
        
        \begin{equation}
            NovaEstimativa \leftarrow AntigaEstimativa + TamanhoPasso \Big[ Objetivo - AntigaEstimativa \Big]
        \end{equation}
        
        \begin{equation}
        \begin{split}
            Q_{n+1} & \ \dot{=} \ Q_n + \alpha \Big[ R_n - Q_n \Big] \\
            & = \ (1 - \alpha)^n Q_1 + \sum_{i=1}^{n} \alpha (1 - \alpha)^{n - i} R_i
        \end{split}
        \end{equation}
        
    \subsection{Escolha dos valores iniciais}
        
        Obs.: citar a diferença entre problemas estacionários e não-estacionários
    
\end{document}
