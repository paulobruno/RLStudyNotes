\documentclass{article}
\usepackage[utf8]{inputenc}
\usepackage[brazil]{babel}
\usepackage[a4paper, top=2cm, bottom=2.5cm, left=3cm, right=3cm]{geometry}
\usepackage{makecell}

\usepackage{graphicx}
\graphicspath{{figures/}}

\title{Aprendizado por Reforço - Notas de aula}
\author{Paulo Bruno de Sousa Serafim}
\date{Setembro 2019}

\begin{document}

\maketitle

\section{\textit{Multi-armed bandit}}

    \subsection{Descrição informal}
    
    \subsection{Formalização}

\section{Dilema \textit{``Exploration vs. Exploitation''}}

    \subsection{Definições}
    
        \subsection{\textit{``Exploration''} = Prospecção}
        
        \subsection{\textit{``Exploitation''} = Exploração}
    
    \subsection{Estratégias}
    
        \subsubsection{Estratégias $\epsilon$}
        
            \begin{itemize}
                \item $epsilon$-greedy
                \item $epsilon$-first
                \item $epsilon$-decay
            \end{itemize}
        
        \subsubsection{Índice de Gittins}

\section{Função ação-valor}

    \subsection{Definição}
    
    \subsection{Versão incremental}
        
    \subsection{Escolha dos valores iniciais}
        
        Obs.: citar a diferença entre problemas estacionários e não-estacionários
    
\end{document}
