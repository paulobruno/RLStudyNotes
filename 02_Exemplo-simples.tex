\documentclass{article}
\usepackage[utf8]{inputenc}
\usepackage[brazil]{babel}
\usepackage[a4paper, top=2cm, bottom=2.5cm, left=3cm, right=3cm]{geometry}
\usepackage{makecell}

\usepackage{pgf}
\usepackage{tikz}
\usetikzlibrary{arrows,automata}

\usepackage{graphicx}
\graphicspath{{figures/}}

\title{Aprendizado por Reforço - Notas de aula}
\author{Paulo Bruno de Sousa Serafim}
\date{Setembro 2019}

\begin{document}

\maketitle

\section{\textit{Multi-armed bandit}}

    \subsection{Descrição informal}
    
    \subsection{Formalização}

\section{Dilema \textit{``Exploration vs. Exploitation''}}

    \subsection{Definições}
    
        \subsection{\textit{``Exploration''} = Prospecção}
        
        \subsection{\textit{``Exploitation''} = Exploração}
    
    \subsection{Estratégias}
    
        \subsubsection{Estratégias $\epsilon$}
        
            \begin{itemize}
                \item $epsilon$-greedy
                \item $epsilon$-first
                \item $epsilon$-decay
            \end{itemize}
        
        \subsubsection{Índice de Gittins}

\section{Função ação-valor}

    \begin{center}
    \begin{tikzpicture}[->,>=stealth',shorten >=1pt,auto,node distance=2.8cm, semithick]
      \tikzstyle{initial-state}=[fill=green,shape=circle,draw=none,text=white]
      \tikzstyle{second-state}=[fill=red,shape=circle,draw=none,text=white]
      \tikzstyle{third-state}=[fill=blue,shape=circle,draw=none,text=white]
        
        \node[initial-state] (A) {$s_1$};
        \node[second-state]  (B) [below of=A, xshift=-4.5cm] {$s_2$};
        \node[second-state]  (C) [below of=A, xshift=-1.5cm] {$s_2$};
        \node[second-state]  (D) [below of=A, xshift= 1.5cm] {$s_2$};
        \node[third-state]   (E) [below of=A, xshift= 4.5cm] {$s_3$};
        
        \path 
            (A) edge[bend right=5] node[left] {$a_1, p_{1}$} (B)
                edge[bend right=5] node[left] {$a_1, p_{2}$} (C)
                edge[bend right=5] node[left] {$a_2, p_{3}$} (D)
                edge[bend right=5] node[left] {$a_3, p_{4}$} (E)
            (B) edge[bend right=5] node[right] {$r_1$} (A)
            (C) edge[bend right=5] node[right] {$r_2$} (A)
            (D) edge[bend right=5] node[right] {$r_3$} (A)
            (E) edge[bend right=5] node[right] {$r_4$} (A);
    \end{tikzpicture}
    \end{center}

    \subsection{Definição}
    
    \subsection{Versão incremental}
        
    \subsection{Escolha dos valores iniciais}
        
        Obs.: citar a diferença entre problemas estacionários e não-estacionários
    
\end{document}
